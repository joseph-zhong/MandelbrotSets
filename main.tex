\documentclass[letterpaper]{article}
\usepackage[T1]{fontenc}
\usepackage{titling, amsmath, amssymb, url, alltt, float, enumitem, graphicx, multicols}
\setlength{\droptitle}{-5em} 
\title{Exploring Dynamic Parallelism in CUDA C with Mandelbrot Sets}
\author{Joseph Zhong
\\ josephz@cs.washington.edu
\\ University of Washington}
\usepackage[margin=0.75in]{geometry}

\setlist{noitemsep}
\graphicspath{images}

\newcommand{\ind}{\hspace{1cm}} 
\newcommand{\m}[1]{\mathbf{#1}}
\newcommand{\x}{\m{x}}
\newcommand{\bigC}{\mathbf{C}}
\newcommand{\bigX}{\mathbf{X}}
\newcommand{\bigI}{\mathbf{I}}

\begin{document}
\maketitle
\begin{multicols}{2}

\section{Introduction}

Mandelbrot Sets are a well studied mathematical concept. Mathematically, the set
  of complex numbers is formally defined as the set of complex value $c$ for 
  which the quadratic $z_{n+1} = z_n^2+c$ remains bounded for a large number of 
  iterations $n$ where $z_0 = 0$. 
Visually, we can actually create a graph of the Mandelbrot set by plotting the
  points $c$ belonging in the Mandelbrot set $M$. As a simple example, we could
  color the values of $c$ which belong in the set one color, and those which
  don't a different color. 
In the context of graphically representing Mandelbrot sets in through computer 
  generated graphics, we can color the points of the set based on the behavior
  of the specific algorithm used to determine whether a point belongs in the set. 
In this report we show examples of  



\end{document}

